% !TEX root = ../MasterThesis.tex

\clearpage

\chapter*{謝辞} \label{sec:Acknowledgement}
本研究を行うにあたり、多くの方々に多大なるご支援とご協力を頂きました。この場を借りて厚く感謝申し上げます。

特に指導教員である末原大幹助教には、日々の研究のことから進路に至るまでとても手厚い助言やサポートをして頂き、感謝の念に堪えません。研究生活のなかで多くの失敗を繰り返してしまいご迷惑をおかけしましたが、毎日の的確な指導と有益なアドバイスのおかげで研究生活を挫けることなく過ごすことができました。多くの失敗については深く自責すると共に、必ず今後の糧としていきたいと考えております。特に、DESYへの出張の際には初めての海外で戸惑いも大きい中、研究機関での活動を無事にやり遂げることができました。また、川越清以教授にはゼミの指導から日々の研究のアドバイス、そして韓国への合宿の参加など多くの面でお世話になりました。韓国での合宿は私の中で海外の学生たちと触れ合う良い機会となり、研究に対する刺激を感じる経験でした。そして何より、末原助教と川越教授には、博士課程への進学を希望した際に検出器と加速器の2つの道を提案していただいた上、相談にも乗ってくださり心より感謝いたします。先の未来について見通しがはっきりせず、考えが定まらなかった私に確かな道標を差し出してくれました。結果として加速器へ進むことを決意いたしましたが、全てはお二人の提案のおかげです。まだまだ実力不足ではありますが、ご2人のご恩に答えるためにもこれから益々精進していきたいと考えております。吉岡准教授におかれましては、博士課程への進学の際の相談や、ILC関連の会議の運営などで大変お世話になりました。東城順治教授には、 学部生の時の研究室訪問での研究室紹介の際にお話を伺い、本研究室を志すきっかけをいただいた他、学部4年の素粒子実験の講義や修士課程での集中講義の際に関わりを持たせて頂き、感謝申し上げます。音野瑛俊助教にはEBESでの実験の解析ミーティングにおいて助言頂きました。森津学助教には、実験のTAの際に一緒に実験を進めてさせて頂き、多くのご支援をいただきました。実験や実験装置に関する深い見識を伺い、TAとしても大きく成長することができたと考えております。山中隆志助教には、居室に毎日来られて研究している姿に日々刺激を受けたと共に、研究の相談をした際には丁寧に回答して頂きました。細川律也学術研究員には、食事の際や研究の合間に多くの会話をしていただいたほか、学振やスライドの草案を添削いただき、感謝いたします。小川真治特別研究員におかれましては、ともに研究室で過ごしたのは短い間ではありましたが、私が相談した際には丁寧に教えて頂きました。水野貴裕学術研究員には、ILCのソフトウェアミーティングでお世話になるとともに、ILCについて相談した際には丁寧にお答え頂き、感謝いたします。重松さおり氏には、日々の事務作業に加えて、特に海外への出張の際に大変お世話になりました。お忙しい中ご迷惑をおかけすることもありましたが、的確な対応のもとでアドバイスをくださり、感謝の念に堪えません。

野口恭平氏には本論文の執筆から、ミーティング係での仕事などとても多岐にわたる場面でお世話になりました。研究に対する真摯な態度は博士課程の学生としての模範にさせて頂きたいと感じております。竹内佑甫氏にはオンライン飲み会で楽しくお話しさせて頂くとともに、学振の書き方について非常に参考になる資料やアドバイスを頂くことができました。学振では非常に近い内容であることもあって、的確なご指導を賜ることができました。また、すでにご卒業された先輩方にも大変お世話になりました。特に、同じくILC計画に参加していた後藤輝一氏には深層学習への興味を持つきっかけを頂き、また研究を自発的に進める姿勢も大変感銘を受けました。そして、久原真美氏にはILCでの研究の仕方から、ビームテスト実験での活動など、多くの知見を賜りました。感謝申し上げます。同輩の尾上友紀氏や宮本佳門氏、谷田征輝氏、樋口義清氏、川上実言氏とは、日々の会話を通じて楽しく研究生活を送る活力になるとともに、一緒に励まし合いながら切磋琢磨し合う良い研究仲間であったと感じています。さらに、後輩の永江君、西原君、星野君、塩谷君、梅林君、山田君、Joey君、David君、土屋君、花田君、吉川君、今村君、水取君とは共に研究生活を送る中で、時には夜遅くまで残って研究活動をする姿に私も刺激を受け、一層身が引き締まる思いを感じました。

その他、学外の方にも多くのご支援をいただきました。DESYおよびIJCLabでの研究ではILC SiWECALグループのRoman Poeschl氏、Adrian Irles氏、Vincent Bondry氏を初めとした関係者の方々の多大なるご厚意に預かりました。特に、Roman Poeshl氏と奥川悠元氏にはフランスでの生活について様々なアドバイスを頂き、無事に出張をやり遂げることができたと感じております。Adrian Irles氏、Vincent Boudry氏、Stephane Callier氏には、英語が不慣れな私にも親切に話して頂き、大変有り難く感じました。また、EBES実験では、石川氏や宮原氏、田窪氏、坂木氏、吉原氏を始めとした多くの方々のご協力を頂きました。KEKでの実験においても、解析ミーティングにおいても多くの有用なアドバイスを頂き、たくさんの学びを得ることができたことに感謝いたします。

最後に、私の研究生活を力強く応援し、手厚いサポートをしてくれた家族や友人へ感謝の気持ちを示し、この論文の結びといたします。