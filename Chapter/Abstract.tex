% !TEX root = ../MasterThesis.tex

\ifabstract
 \maketitle
\fi
\begin{center}
{\huge 概要}\\
\end{center}


%----- 物理に関する概要
%20世紀の数々の素粒子の発見を始め、素粒子物理の基礎理論である標準理論Standard Model, SM)は実験的事実の多くを正確に予測してきた。

%今後の素粒子物理の進展に向けて、超対称性理論(Supersymmetry,SUSY)や大統一理論(Grand Unification Theory, GUT)のような理論は新物理の予測を可能にし、将来にわたる素粒子実験の一つの指針として機能している一方で、現状のLarge Hadron Collider(LHC)における測定ではこれらの理論で予測されるような標準理論からのずれは確認されておらず、広い視野を持って新たな可能性を探る必要が改めて重要となっている。その実現に向けて、LHCにおいて発見が難しいと考えられる新物理に感度を持った電子陽電子ヒッグスファクトリーと反応が稀な新物理を探索する固定標的実験の双方を行い、未知の事象を探索することが計画されている。%しかし、未だ物理学上の未解明な現象や標準理論と反する観測結果が存在し、その解明のために標準理論を超える更なる理論の確立が待たれている。SUSYやTeVスケールにおける物理のような理論は新物理の予測を可能にし、将来にわたる素粒子実験の一つの指針として機能している。一方で、現状のLarge Hadron Collider(LHC)における測定ではこれらの理論で予測されるような標準理論からのずれは確認されていない。そのために、広い視野を持って新たな可能性を探る必要が改めて重要となっている。その実現に向けて、LHCにおいて発見が難しいと考えられる新物理に感度を持った電子陽電子ヒッグスファクトリーと反応が稀な新物理を探索する固定標的実験の双方を行い、未知の事象を探索することを目指す。
%現状のLarge Hadron Collider(LHC)における測定を振り返ると、未知の理論により予測されるような標準理論からのずれは確認されておらず、広い視野を持って新たな可能性を探る必要が改めて重要となっている。その実現に向けて、LHCにおいて発見が難しいと考えられる新物理に感度を持った電子陽電子ヒッグスファクトリーと反応が稀な新物理を探索する固定標的実験の双方を行い、未知の事象を探索することが計画されている。
ヒッグスの精密測定とTeVスケールの新物理探索を目的とした電子陽電子ヒッグスファクトリーと、同じく電子陽電子を用いたGeVスケールの弱結合新物理を探索する固定標的実験は、素粒子標準模型の枠に含まれない物理を相補的に探索し、宇宙の謎に迫る素粒子実験の次期計画である。
%カロリメータ技術は入射してきた粒子のエネルギーを正確に測定するために必要不可欠な要素となっている。カロリメータ技術の中には共通して利用することが可能なものも含まれる。
本研究ではこれらの実験に共通する重要な測定技術であり、入射粒子のエネルギーを正確に測定するカロリメータ技術に着目して、その実装と各実験での応用を検討した。

電子陽電子ヒッグスファクトリーとして、東北地方・北上山地に建設が予定されているのが国際リニアコライダー(International Linear Collider, ILC)である。%本研究ではその衝突点に置かれるInternational Large Detector(ILD)に搭載される電磁カロリメータ(Electromagnetic Calorimeter, Ecal)について着目する。ILCに搭載される予定の検出器にはInternational Large Detector(ILD)とSilicon Detector(SiD)があり、本研究ではILDに搭載される電磁カロリメータについて着目する。
%SM粒子とはスピンが1/2異なる超対称性粒子を仮定するSUSYやより高エネルギーにおける新物理が期待されるTeVスケール物理といった新たな理論が構築されている。
%特に、2012年にその存在が明らかとなったヒッグス粒子については正確な質量や崩壊分岐比など、未解明の性質が多く、実験的観測による新物理発見が大いに期待されている。その精密測定を行うことを主目的としているのが、International Linear Collider(ILC, 国際リニアコライダー)であり、岩手県北上山地に建設予定の電子陽電子コライダーとして研究が進められている。
%開始運転時の重心エネルギーが250\ GeVとして設定され、これはヒッグスファクトリーとして運転するために十分なエネルギーとなる。
%本研究はその最先端の実験にも利用可能な高度な実験装置および解析手法の開発を目的としている。
   粒子の相互作用により生じるハドロン粒子束から成るジェットは、新物理に関わるヒッグス分岐比を測定する上で重要な情報であり、その正確なジェットエネルギーの測定が求められる。Particle Flow Algorithm (PFA)は異なる検出器ごとに各粒子を分離し、分解能を高めるための有力な手法として開発が行われており、十分な性能を持ったPFAの開発には微細分割したカロリメータと粒子を分離するためのアルゴリズムが必要となる。GravNetは欧州原子核研究共同機構(CERN)のLHC/CMS実験に導入されるHigh Granularity CALorimeter (HGCAL)のために開発された深層学習ネットワークである。PFAの過程の中で重要な部分であるカロリメータクラスタリングの向上を目的として研究されており、従来の手法を上回るクラスタリング性能が示されている。CMSのHGCALはILCで用いられるカロリメータとの共通点が多い。そのため、本研究ではILCの測定器シミュレーションにおいて近接する2光子事象の分離能力を検証した。その結果、GravNetが2つの光子を効率的に分離できることを示した。
   %Particle Flow Algorithm (PFA)は各粒子を分離し、個別にジェットエネルギーを測定することで分解能を高める有力な手法として開発が行われている。しかし、現在のILCで採用されているものは人の手によるカッティング手法を元にしたアルゴリズムとなっており、シミュレーションの結果から更なる改善が見込まれている。そのために本研究で採用したのが機械学習技術の一つである深層学習である。特に、グラフニューラルネットワークは検出器技術との親和性が高く、最適となるネットワークを用いることで人力による解析よりも優れた結果をもたらすことが期待されている。その中でも、Gravnet技術はCMSにおけるhigh granularity calorimeterのために開発されたシャワー識別ネットワークであり、Object Condensationと呼ばれる損失関数と組み合わせることでジェットエネルギー分解能の向上に貢献しうることが示されている。本研究ではこの手法のILCへの適用と評価を行う。
  
  また、電子陽電子固定標的実験としては、KEK(高エネルギー加速器研究機構)入射器ビームダンプにおいてAxion Like Particles (ALPs)探索を探索するEBES (Electron Beam dump Experiment at SY3)実験に参加した。この実験では電子・陽電子ビームを固定標的に照射し、光子が原子核と相互作用して擬スカラーメソンが生じるPrimakoff効果によるALPsの生成を通じて生成された光子ペアをカロリメータにより測定する。この際、ALPsへの崩壊は稀であるため中性子やミューオンといったバックグラウンドイベントが測定の障害となりうる。従って、他のバックグラウンドイベントと明確に分離するためにビーム衝突により生じた粒子のエネルギーをカロリメータで正確に測定することが必要である。本研究ではその実現のために、KEKにおいて2つのビームテストを行った。一つはKEK Linacで行われたバックグラウンド測定、もう一つはKEK AR-TBで行われた鉛ガラス検出器性能評価実験である。本研究ではその解析結果を示すとともに、本実験に向けた今後のカロリメータ技術の応用について考察する。
 

%----- COMET実験に関する概要
%xxx実験は、・・・。

%----- 本研究に関する概要
%本研究では、・・・。
%また、・・・を行った。
%その結果、・・・を示した。

%----- 今後の課題に関する概要
%この結果をもとに、・・・。
