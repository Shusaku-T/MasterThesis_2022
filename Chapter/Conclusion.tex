% !TEX root = ../MasterThesis.tex

\chapter{まとめおよび今後の展望} \label{sec:Conclusion}

% xxx実験に関するまとめ
\section{まとめ}
本研究では、電子陽電子実験であるILC計画およびEBES実験において、主要な技術の一つであるカロリメトリーに関する開発および実装を行った。光子および電子のエネルギーを測定し、目的とする物理を探索する上で重要となるカロリメータシステムに関してさらなる高度な技術開発が求められている。EBES実験において鉛ガラス検出器はALPsから崩壊する2光子を探索する上で重要なカロリメータであり、その性能は稀な反応を捉えるために十分なものが要求される。そして、ILCへの実装に向けて開発が行われているPFAはジェットエネルギー分解能を改善するためのアルゴリズムとしてさらなる高精度が期待されている。本研究ではこれらのカロリメータ技術について、ソフトウェアおよびハードウェアの両面から探索を行ったものである。

 まず、近年の解析技術の発展に伴って注目が集められている深層学習技術であるGravNetおよびObject Condensationと呼ばれる技術を用いて、微細分割カロリメータ内のクラスターを同定するアルゴリズムを開発した。実装は深層学習のためのライブラリが豊富に提供されているPyTorchを用いて行い、入力データとしてILD検出器シミュレーションのデータを使用した。2つの近接した光子をILDカロリメータ内に入射させ、それぞれのクラスターが識別できているか評価を行った。解析の結果、角度$\SI{0.5}{rad}$において99.56\%の精度で2つのクラスターが識別できていることを確認できた一方で角度を$\SI{0.1}{rad}$に狭めると、96.08\%まで低下した。

続いて、将来のALPs探索実験であるEBES実験に関して、本実験に向けた準備としてバックグラウンド測定と鉛ガラス検出器性能測定を行った。バックグラウンド測定ではKEK・Linacを用いてビームダンプへの照射試験を行い、鉛ガラス検出器応答を調べた。測定の結果、ビームパイプ内でビームロスが生じ、制動放射によって生成された光子がバックグラウンドとして検出器へ入射している可能性が示唆された。この可能性は引き続き調査し、定量的に評価する必要がある。

また、KEKに新設されたPF/ARテストビームラインにおいて鉛ガラス検出器のエネルギー較正およびエネルギー分解能評価を行った。それぞれの検出器に対して、ビームエネルギーに\SI{0.1128}{GeV}の補正を行うと、ADCとビームエネルギーの間に線形性が見られた。また、1つの検出器に対して何通りかのフィッティングを行い、エネルギー分解能として$\sigma_E/E\approx7.3\%/\sqrt{E}$が得られた。この数値は以前測定された値より悪化しているものの、EBES実験においてALPを観測するために必要だと概算された分解能$15\%/\sqrt{E}$を達成している。


\section{今後の展望}
GravNetによるカロリメトリー技術についてはさらに定量的かつ詳細な評価を行うことが今後の性能向上に向けて重要となる。識別のできなかったヒットのエネルギーや複数のMC粒子が混在したイベントにおける性能などを指標を用いて明示する。実際の検出器イベントはシミュレーションよりもずっと複雑になりうるため、粒子の運動量や種類など、あらゆる要素を考慮に入れたシミュレーションの実施とそのGravNetの適用も必要な課題である。今回はカロリメータ内の1つのセルに複数のヒットが入射することで生じるConfusionと呼ばれる効果は少なかったが、さらに多くのMC粒子が入射することでこの影響が大きくなり識別が困難になることが予期される。そのための対策を講じることも今後の課題の1つである。

EBES実験に関しては、今回行った測定を基にして本実験に向けた詳細なシミュレーション解析や実験セットアップの改善などが予定される。並行してEBES実験の導入に向けたSiW ECALの開発および性能評価も引き続き行われる。

%ジェットのエネルギーを変化させた際の識別能力の比較や、MC粒子の種類を変化させた時の

